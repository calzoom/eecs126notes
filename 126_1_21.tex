\documentclass[11pt]{scrartcl}
\usepackage[header, margin, koma]{japjot}

\pagestyle{fancy}
\fancyhf{}
\fancyhead[l]{EECS 126 Notes}
\fancyhead[r]{Japjot Singh}
\cfoot{\thepage}

\begin{document}
\title{\Large EECS 126 Notes}
\author{\large Japjot Singh}
\date{\large\today}

\maketitle

Course notes for EECS 126 taken in Spring 2020. 

\tableofcontents
\newpage

\section{Tuesday, January 21} 
Understand problem as an "experiment" and then solve it using tools in your
skillset: combinatorics, calculus, common sense.
\subsection{Fundamentals}
\begin{definition}
	Sample Space $\Omega$ of an experiment is the set of all oucomes of
the experiment. 
\end{definition}

\begin{example}
	Your experiment is 2 fair coins $\Omega = \{HH, HT, TH, TT\}$
these outcomes (base outcomes) are \textbf{mutually exclusive (ME) and
collectively exhaustive (CE)} 
\end{example} 

\begin{example}
Toss a coin till the first "Heads" $\Omega = \{H, TH, TTH,
\ldots\};|\Omega|= \infty$
\end{example} 

\begin{example}
Waiting  at the bus-stop for next bus $\Omega = (0, T)$
\end{example} 

Visual 1 - We have the experiment which produces outcomes, once you have the
outcome space the next definition is the definition of events

\begin{definition}[Events]
Allowable subsets of $\Omega$ (collections of
outcomes) 
\end{definition}

\begin{example}
Get at least 1 Head in experiment 1, $\{HH, HT, TH\}$,
$p=\frac{3}{4}$ 
\end{example} 

\textbf{Defining events carefully is the key to tackling many tough problems.}

\begin{example}
Ex 2.2: Get an even number of tosses
\end{example} 

\begin{example}
Ex 2.3: Waiting time < 5 min
\end{example} 

\begin{definition}[Probability Space]
	 A \textbf{probability space} $(\Omega, \mathcal{F} , \mathcal{P} )$ is a mathematical construct
to model "experiments" and has 3 components: \begin{enumerate}
  \item $\Om$ is the set of all possible outcomes
  \item $\mathcal{F}$ set of all events (composition of outcomes), where each event is a set containing
    0 or more base outcomes, \textbf{$\emptyset$ is a base outcome} where $\pr(\emptyset) = 0$.
      $\mathcal{F} $ is intuitively a powerset
    (i.e. for the experiment in example 1.1 $\mathcal{F}=\{\emptyset, \{H,H\}, \{H, T\},
    \ldots\}$).
    \item $\mathcal{P}$ is the proability measure which assigns a number in $[0, 1]$ to each event in $\mathcal{F}$.
\end{enumerate}
\end{definition}

\textbf{Base oucomes must be ME and CE} that is when writing out $\Om$ as a collection of all the base outcomes, they should be the most simplified components.

\subsection{Axioms of Probability (Kolmogov)}
What properties do we need the probability measure $\mathcal{P} $ to satisfy?
\begin{enumerate}
    \item $\pr(\emptyset)=0$ 
    \item $\pr(\Omega)=1$, really just a normalization
    \item $\pr(A_1\cup A_2\cup\ldots)=P(A_1)+P(A_2)+\ldots$ for disjoint (ME)
      events $A_1,A_2,\ldots$
\end{enumerate}
for disjoint $\pr(\bigcup_{i=1}^n A_{i} )= \sum_{i=1}^n \pr(A_{i} )$ 
\subsection{Fundamental facts about probability} 
\begin{enumerate}
    \item $\pr(A^{c}) = 1 - \pr(A)$ 
    \item $\pr(A\cup B) =  \pr(A)+\pr(B)-\pr(A\cap B)$ \textbf{Vis2 Venn
      Diagram}  
      \item Union-bound $\pr(A_1\cup A_2\cup\ldots\cup A_{n} )\leq \sum_{i=1}
        ^{n} \pr(A_{i} )$
      \item Inclusion-Exclusion, a generalized version of number 2
\end{enumerate}

\begin{theorem}[Inclusion-Exclusion]
	$\pr(A_1\cup A_2\cup\ldots\cup A_{n} )= \sum_{k=1}^n \sum_{1\leq i_1} \sum_{\leq i_2}\ldots\sum_{i_{k} \leq
		n} (-1)^{i+1}$
\end{theorem}
\begin{proof}
	here
\end{proof}

\subsection{Discrete Probability} 
\[
\pr(A) = \sum_{\omega \in  A} \pr(\omega)
\] 
In a uniform sample space, all the outcomes are equally likely so then \[
  \pr(A) = \frac{|A|}{|\Omega|}
\]
\subsection{Conditional Probability} 
Similar to events, conditioning on the right event will bail you out of tricky problems.
\begin{definition}
  $P(A|B) := \pr(\text{Event A given that Event B has occured})$ 
\end{definition} 
Thus for any event $A$ if $\pr(B)\neq 0$, \[
\pr(A|B) = \frac{\pr(A\cap B)}{\pr(B)}\] 
\begin{example}
  Consider 2 six-sided dice. Let $A$ be the event that the first dice rolls is
  a 6. Let  $B$ be the event the sum of the two dice is 7. Then \[
    \pr(A|B) = \frac{\pr(A\cap B)}{\pr(B)} = \frac{\pr(\{6,1\})}{\pr(\{6,1\},
    \{5,2\},\ldots,\{1,6\})}
  \]
  Similarly $\pr(A|\text{sum is 11})=\frac{1}{2}$
\end{example}
When conditioning on $B$,  $B$ becomes to new  $\Om$. 

\subsection{Product (Multiplication) Rule} 
\subsection{Total Probability}
\subsection{Bayes' Theorem} 
\newpage
\section{Thursday, January 23}
\subsection{Announcement}
Readings B\&T ch1 and 2, HW 1 due next wednesday one minute before midnight 

\subsection{Birthday Paradox} 
Assuming a group of $n$ individuals whose birth dates are distributed uniformly
at random. Given $k=365$ days in the year what is the probability that at least
2 people in the group share the same birthday. Our sample space is the consists
  of each possible set of assignments of birth dates to the  $n$ students in
  the class. Since there are 365 posisble days for each of the  $n$ students in
  the group $|\Om| = k^{n} = 365^{n} $. Now we can define our event of
  interest, $A$, that at least 2 people have the same birthday. Since this is
  a hard event to work with we can look at the complement $A^{c} $ the event
  that no two people share a birth date. We can reach the solution with
  a counting argument
  \[
    \pr(A^{c}) = \frac{|A^{c}|}{|\Om|} = \frac{365 * 364 *\ldots*(365
    - (n-1))}{365^{n} }
  \]
  or with a probabilistic argument using the chain rule \[
    \pr(A^{c} ) = 1(1-\frac{1}{k})(1-\frac{2}{k})\cdots(1 - \frac{n-1}{k})
  \]
  the latter expression can be approximated using Taylor Series which say
  $e^{x} \approx 1 + x $ for $|x|<<1$.  \[
    \pr(A^{c} )\approx 1\cdot e^{-\frac{1}{k}} \cdot e^{-\frac{2}{k}} \cdots
    e^{-\frac{n-1}{k}} 
  \] thus $\pr(A)=1-\pr(A^{c} )\approx 1 - e^{-\frac{n^2}{2k}} $ 

\subsection{Bayes Rule False Positive Problem} 
Suppoes there is a new test for a rare disease. 
\begin{itemize}
  \item If a person has the disease, test positive with $p=0.95$
     \item If person does not have disease, test negative with $p=0.95$ 
       \item Random person has the diseaes with $p=0.001$
\end{itemize}
Suppose a person tested positive, what is the probability that person has the
disease. Let $A$ be the event has disease and  $B$ be the event test positive
then by applying Bayes Rule directly  \[
  \pr(A|B) = \frac{(0.95)(0.001)}{(0.95)(0.001) + (0.999)(0.05)} = 0.1875
\] the factor heavily contributing to this number is the prior, how rare the
diesase is in the first place. 
\subsection{Independence} 
\begin{definition}
  Two events are independent if the occurence of one provides \textbf{no
  information} about the occurence of the other (i.e. $\pr(A|B) = \pr(A)$). 
\end{definition}
\textbf{insert vis1}  
Indepedence can also be written as \[
  \pr(A\cap B) = \pr(A)\pr(B)
\] 
\textbf{Note:} Disjoint events are \textbf{not} Independent. Events $A$ and
$B$ are disjoint if and only if $\pr(A\cap B) = 0 \implies  \pr(A)=0 \lor
\pr(B)=0$. Thus since Base outcomes of a random experiment are disjoint (ME)
and have non-zero probabilities they \textbf{must be dependent}.

\subsubsection{Conditional Independence} 
\[
\pr(A\cap B|C) = \pr(A|C) \cdot \pr(B|C)
\] 
Note that \begin{itemize}
  \item Dependent events can be conditionally independent
    \item Independent events can be conditionally dependent
\end{itemize}
\begin{example}
  Consider 2 indistinguishable coins: one is two-tailed and the other is
  two-headed. You pick on of the 2 coins at random and flip it twice. \\\\
  Let $H_{i}$ be the event that the $i^{th} $ flip is a Head ($i=1,2$). By
  itself $\pr(H_1)=\pr(H_2) = \frac{1}{2}$ and $\pr(H_2|H_1) = 1 \neq
  \pr(H_2) = 1/2$. Furthermore, $\pr(H_1\cap H_2|A) = \pr(H_1|A)\pr(H_2|A\cap
  H_1)= \pr(H_1|A)\pr(H_2|A)$ which by definition tells us that $H_1,H_2$ are
  conditionally independent given $A$.
\end{example}

\subsubsection{Independence of a collection of events}
For all possible subsets of your events $A_{1:n} $, each subset must be
independent that is
\[
  \pr(\bigcap_{i\in S}A_{i}) = \prod_{i\in S}(\pr(A_{i} )), \forall S
\] where $S$ is any subset of the collection of events. Pairwise independence
\textbf{does not imply} Joint independence of 3 or more events. 
\end{document}
